\documentclass[slideopt,A4,showboxes,svgnames]{beamer}

%% list of packages here
\usepackage[absolute,showboxes,overlay]{textpos}
\usepackage{booktabs}
\usepackage{mathtools}
\usepackage{amsmath}
\usepackage{amssymb}
\usepackage{pifont}
\usepackage{multicol}
\usepackage{multirow}
\usepackage{array, makecell}
\usepackage{blkarray}
\usepackage[linesnumbered,ruled,vlined]{algorithm2e}
\usepackage[backend=biber, style=authoryear]{biblatex}
\renewcommand*{\nameyeardelim}{\addcomma\addspace}
\addbibresource{../references.bib}

\usepackage{theme/beamerthemeinria}
\input{mathdef}
\input{theme/style}

\title[Monte-Carlo Graph Search]{Monte-Carlo Graph Search:}
\subtitle{the Value of Merging Similar States}
\date[date]{date}
\author[Edouard Leurent]{\textbf{Edouard Leurent}\inst{1,2},\\
	Odalric-Ambrym Maillard\inst{1}}
\institute{
	\inst{1} Univ. Lille, Inria, CNRS, \\ ~Centrale Lille, UMR 9189 – CRIStAL,\\
	\inst{2} Renault Group}

\begin{document}

\begin{frame}
    \titlepage
\end{frame}

\begin{frame}{Motivation}
\begin{alertblock}{Monte-Carlo Tree Search algorithms}
\begin{itemize}[<+->]
	\item rely on a \alert{tree structure} to represent their value estimates.
	\begin{center}
		\includegraphics[trim={1.8cm 2.2cm 1.9cm 2.7cm}, clip, width=0.38\linewidth]{img/tree_simple}
	\end{center}
	\item performance {\green independent} of the {\red size $S$ of the state space}%, but with the branching factor $K$ and planning horizon $H$ / budget $n$.
\begin{center}
	\begin{tabular}{lcc}
		Tabular RL & (\texttt{UCBVI}) & $\sqrt{H{\red S}{\green A}n}$ \\
		MCTS & (\texttt{OPD}) & $n^{-\log\frac{1}{\gamma}/\log {\green A}}$
	\end{tabular}
\end{center}
\end{itemize}
\end{alertblock} 
\end{frame}

\begin{frame}{Limitation}
	\begin{itemize}
		\item There can be several paths to the same state {\orange $s$}
		\begin{center}
			\includegraphics<1>[trim={3cm 8cm 3cm 4cm}, clip, width=0.85\linewidth]{img/paths_1}%
			\includegraphics<2>[trim={3cm 8cm 3cm 4cm}, clip, width=0.85\linewidth]{img/paths_2}%
			\includegraphics<3>[trim={3cm 8cm 3cm 4cm}, clip, width=0.85\linewidth]{img/paths_3}%
			\includegraphics<4-6>[trim={3cm 8cm 3cm 4cm}, clip, width=0.85\linewidth]{img/paths_4}%
		\end{center}
		\item<5-6>  {\orange $s$} is represented several times in the tree
		\item<6> {\red No information is shared} between these paths
	\end{itemize}	
\end{frame}

\begin{frame}{Motivating example 1/2}
\begin{alertblock}{Not accounting for state similarity hinders exploration}
	\pause
	Sparse gridworld: reward of 0 everywhere
	\pause
	\begin{itemize}
		\item[\incarrow] \alert{Uniform planning} in the space of sequences of actions %  (UCT, OPD, OLOP, etc.) which seems reasonable
	\end{itemize}
	\begin{center}
		\includegraphics[trim={0 1cm 0 1.5cm},clip,width=0.7\linewidth]{../img/tree_OPD}

		OPD, budget of $n=5460$ moves
	\end{center}
\end{alertblock}
\end{frame}

\begin{frame}{Motivating example 2/2}
\begin{alertblock}{Concentration}
\begin{itemize}[<+->]
\item Does not lead to uniform exploration of the state space
% but that does not lead to uniform exploration in the state space, because many overlaps between the explored paths
% it spends a lot of time to move mack and forth to states already visited through other trajectories
\item 2D random walk $\sim$ Rayleigh distribution
$P(d) = \frac{2d}{H}e^{-\frac{-d^2}{H}}$
% similar to random walk: if you sample moves with a uniform distribution, you will not explore far and concentrate around the initial position. Probability to reach a distance d is (1/4)^d

\begin{center}
	\includegraphics[width=0.6\linewidth]{../img/occupations_OPD}
	
	budget of $5460$ samples, maximum distance $d=6$
\end{center}
\end{itemize}
\end{alertblock}
\end{frame}

\begin{frame}{Goal}
\begin{exampleblock}{Better exploit this wasted information}
	\begin{itemize}
		\item<2-4> By \alert{merging} similar states \only<3-4>{into a {\green graph}}
	\end{itemize}
	\begin{center}
		\includegraphics<1>[trim={1.8cm 2.2cm 1.9cm 2.7cm}, clip, width=0.55\linewidth]{img/tree_simple}%
		\includegraphics<2>[trim={1.8cm 2.2cm 1.9cm 2.7cm}, clip, width=0.55\linewidth]{img/tree_merging}%
		\includegraphics<3>[trim={1.8cm 2.2cm 1.9cm 2.7cm}, clip, width=0.55\linewidth]{img/tree_merged}%
		\includegraphics<4>[trim={1.8cm 1.2cm 1.9cm 0.8cm}, clip, width=0.55\linewidth]{img/graph_simple}%
	\end{center}
\end{exampleblock}
\end{frame}

\begin{frame}{Questions}
\begin{center}
	\includegraphics[trim={1.8cm 2.2cm 1.9cm 2.7cm}, clip,width=0.46\linewidth]{img/tree_simple}
	\includegraphics[trim={1.8cm 1.2cm 1.9cm 0.8cm}, clip,width=0.46\linewidth]{img/graph_simple}
\end{center}
\begin{itemize}
	\item How to \alert{adapt} MCTS algorithms to work on graphs?
	\item Can we \alert{quantify} the benefit of using graphs over trees?
\end{itemize}
\end{frame}

\begin{frame}{MCTS for Deterministic MDPs}
\begin{center}
	\includegraphics<1-3>[trim={1.8cm 1.5cm 1.9cm 2.7cm}, clip,width=0.35\linewidth]{../img/tree_1}%
	\includegraphics<4>[trim={1.8cm 1.5cm 1.9cm 2.7cm}, clip,width=0.35\linewidth]{img/tree_sample}%
	\includegraphics<5>[trim={1.8cm 1.5cm 1.9cm 2.7cm}, clip,width=0.35\linewidth]{img/tree_expand}%
\end{center}
\vspace*{-0.5cm}
\begin{block}{Optimism in the Face of Uncertainty: \texttt{OPD}}
\pause
\begin{enumerate}[<+->]
	\item Build \alert{confidence bounds} $L(a)\leq V(a) \leq U(a)$ on the value $V(a)$ of every sequence of action $a$
	\item Select a candidate for expansion \begin{itemize}
		\item Start from the root, follow \alert{optimistic} actions, down to a leaf $b$
	\end{itemize}
	\item \alert{Expand} the leaf $b\in\ext{T}_n$
\end{enumerate}
\end{block}
\end{frame}


\begin{frame}{On a graph...}

\begin{center}
	\includegraphics<1-2>[trim={1.0cm 0.8cm 1.9cm 0.8cm}, clip,width=0.35\linewidth]{../img/graph_1}%
	\includegraphics<3>[trim={1.0cm 0.8cm 1.9cm 0.8cm}, clip,width=0.35\linewidth]{img/graph_sample}%
	\includegraphics<4-6>[trim={0.8cm 0.8cm 1.9cm 0.8cm}, clip,width=0.35\linewidth]{img/graph_expand}%
\end{center}

\begin{block}{Same principle: \texttt{GBOP-D}}
	\pause
	\begin{enumerate}[<+->]
		\item Build \alert{confidence bounds} $L(s)\leq V(s) \leq U(s)$
		\item Follow \alert{optimistic} actions \textcolor<6>{red}{until an external node $s$ is reached}
		\item \alert{Expand} the external node $s\in\ext{\cG}_n$
		\begin{itemize}
			\item[\incarrow] any state $s$ can {\green only be expanded once}
		\end{itemize}
	\end{enumerate}
\end{block}
\end{frame}

\begin{frame}{How to build the bounds $L,U$?}
$$ V(s) = \sup \sum_{t=0}^{\infty} \gamma^t r_t $$
\begin{itemize}[<+->]
	\item Initialize with trivial bounds
	$$L=0,\qquad U=\frac{1}{1-\gamma}$$
	\item Apply the Bellman operator $B$ at every internal nodes $a$:
		$$V(a) \leq B(U)(a) = \max_{b\in A} r(ab) + \gamma U({ab}) \leq U(a)$$
		$$V(a) \geq B(L)(a) = \max_{b\in A} r(ab) + \gamma L({ab}) \geq L(a)$$
	\item Converges in {\green $d_n$ steps} for trees
	\item Converges in {\red $\infty$ steps} on graphs, when there is a loop
\end{itemize}
\end{frame}

\begin{frame}{Analysis}
Is \texttt{GBOP-D} more efficient than \texttt{OPD}?
\pause
\begin{theorem}[Sample complexity of \texttt{OPD}, \cite{Hren2008optimistic}]
	\[r_n = \tilde{\cO}\left( n^{-\log \frac{1}{\gamma}/{\red \log\kappa}}\right),\]
	where ${\red \kappa}$ is a problem-dependent difficulty measure.
\end{theorem}
\pause
\begin{theorem}[Sample complexity of \texttt{GBOP-D}]
	\[r_n = \tilde{\cO}\left(n^{-\log \frac{1}{\gamma}/{\green \log \kappa_\infty}}\right),\]
	where ${\green \kappa_\infty}$ is a {\green tighter} problem-dependent difficulty measure:
	\[ {\green \kappa_\infty}\leq {\red \kappa} \]
\end{theorem}
\end{frame}

\begin{frame}{Comparing difficulty measures}
\begin{itemize}[<+->]
	\item ${\green \kappa_\infty} = {\red \kappa}$ if the MDP has a tree structure
	\item ${\green \kappa_\infty} < {\red \kappa}$ when trajectories overlap a lot
	\begin{itemize}
		\item actions cancel each-other out (e.g. moving left or right)
		\item actions are commutative (e.g. placing pawns on a board)
	\end{itemize}
\end{itemize}
\pause[\thebeamerpauses]
\begin{exampleblock}{Illustrative example: 3 states, $K>2$ actions}
	\begin{center}
    \includegraphics[trim={0.5cm 0.0cm 0.3cm 0.6cm}, clip, width=0.35\textwidth]{../img/mdp.pdf}
	\end{center}
	 $${\green \kappa_\infty = 1} < {\red \kappa = K-1}$$
\end{exampleblock}
\end{frame}

\begin{frame}{Experiment: sparse gridworld}
Rewards in a ball around $(10, 10)$ of radius $5$, with quadratic decay
\begin{center}
	\includegraphics[trim={1.8cm 0.4cm 1.8cm 0.7cm}, clip, width=0.43\linewidth]{../img/occupations_OPD.pdf}
	\includegraphics[trim={1.8cm 0.4cm 1.8cm 0.7cm}, clip, width=0.43\linewidth]{../img/occupations_GBOP-D.pdf}

	$n = 5640$ samples
\end{center}
\end{frame}

\begin{frame}{Extension to stochastic MDPs}
\begin{alertblock}{What we did for deterministic MDPs\dots}
Use state similarity to \alert{tighten} the bounds $(L,\,U)$ for the value $V$.
\end{alertblock}
\pause
\begin{exampleblock}{\dots we can do too for stochastic MDPs}
\begin{itemize}
	\item The same can be done for any stochastic planning algorithm based on such bounds
	\item We adapt MDP-GapE (\cite{Jonsson2020planning}) to obtain GBOP
\end{itemize}
\end{exampleblock}
\end{frame}


\begin{frame}{Experiment: stochastic gridworld}
Noisy transitions with probability $p=10\%$
\begin{center}
	\includegraphics[trim={1.8cm 0.4cm 1.8cm 0.7cm}, clip, width=0.43\linewidth]{../img/occupations_UCT.pdf}
	\includegraphics[trim={1.8cm 0.4cm 1.8cm 0.7cm}, clip, width=0.43\linewidth]{../img/occupations_GBOP.pdf}
	
	$n = 5640$ samples
\end{center}
\end{frame}

\begin{frame}{Exploration-Exploitation score}
$$S = \sum_{t=1}^n \underbrace{d(s_t, s_0)}_{\text{Exploration}} - \underbrace{d(s_t, s_g)}_{\text{Exploitation}}$$
\begin{center}
	\includegraphics[trim = {1.4cm 0.cm 2cm 1.5cm}, clip, width=0.7\linewidth]{../img/score.pdf}

	$n = 5640$ samples
\end{center}
\end{frame}

\begin{frame}{Sailing Domain (\cite{Vanderbei1996})}
\begin{center}
	\includegraphics[trim = {0.2cm 0.2cm 0.7cm 0.5cm}, clip, width=0.6\linewidth]{../img/simple_regret.pdf}
\end{center}
\pause
\vspace*{-0.5cm}
Effective branching factor $\kappa_e$:
\begin{itemize}
	\item $\kappa_e \approx {\red 3.6}$, for \texttt{BRUE}, \texttt{KL-OLOP}, \texttt{MDP-GapE}, \texttt{UCT}
	\item $\kappa_e \approx {\green 1.2}$ for \texttt{GBOP}, which suggests our results may still hold
\end{itemize}
\end{frame}

\begin{frame}
\centering \LARGE Thank You!
\end{frame}


\end{document}
